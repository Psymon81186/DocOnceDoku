%%
%% Automatically generated file from DocOnce source
%% (https://github.com/hplgit/doconce/)
%%

% #define PREAMBLE

% #ifdef PREAMBLE
%-------------------- begin preamble ----------------------

\documentclass[%
oneside,                 % oneside: electronic viewing, twoside: printing
final,                   % draft: marks overfull hboxes, figures with paths
chapterprefix=true,      % "Chapter" word at beginning of each chapter
open=right,              % start new chapters on odd-numbered pages
10pt]{book}

\listfiles               % print all files needed to compile this document

\usepackage[a4paper]{geometry}

\usepackage{relsize,makeidx,color,setspace,amsmath,amsfonts,amssymb}
\usepackage[table]{xcolor}
\usepackage{bm,microtype}

\usepackage[pdftex]{graphicx}

% user-provided packages: --latex_packages=minted,ngerman,parskip
\usepackage{minted,ngerman,parskip}

% Packages for typesetting blocks of computer code
\usepackage{fancyvrb,framed,moreverb}

% Define colors
\definecolor{orange}{cmyk}{0,0.4,0.8,0.2}
\definecolor{tucorange}{rgb}{1.0,0.64,0}
\definecolor{darkorange}{rgb}{.71,0.21,0.01}
\definecolor{darkgreen}{rgb}{.12,.54,.11}
\definecolor{myteal}{rgb}{.26, .44, .56}
\definecolor{gray}{gray}{0.45}
\definecolor{mediumgray}{gray}{.8}
\definecolor{lightgray}{gray}{.95}
\definecolor{brown}{rgb}{0.54,0.27,0.07}
\definecolor{purple}{rgb}{0.5,0.0,0.5}
\definecolor{darkgray}{gray}{0.25}
\definecolor{darkblue}{rgb}{0,0.08,0.45}
\definecolor{darkblue2}{rgb}{0,0,0.8}
\definecolor{lightred}{rgb}{1.0,0.39,0.28}
\definecolor{lightgreen}{rgb}{0.48,0.99,0.0}
\definecolor{lightblue}{rgb}{0.53,0.81,0.92}
\definecolor{lightblue2}{rgb}{0.3,0.3,1.0}
\definecolor{lightpurple}{rgb}{0.87,0.63,0.87}
\definecolor{lightcyan}{rgb}{0.5,1.0,0.83}

\colorlet{comment_green}{green!50!black}
\colorlet{string_red}{red!60!black}
\colorlet{keyword_pink}{magenta!70!black}
\colorlet{indendifier_green}{green!70!white}

% Backgrounds for code
\definecolor{cbg_gray}{rgb}{.95, .95, .95}
\definecolor{bar_gray}{rgb}{.92, .92, .92}

\definecolor{cbg_yellowgray}{rgb}{.95, .95, .85}
\definecolor{bar_yellowgray}{rgb}{.95, .95, .65}

\colorlet{cbg_yellow2}{yellow!10}
\colorlet{bar_yellow2}{yellow!20}

\definecolor{cbg_yellow1}{rgb}{.98, .98, 0.8}
\definecolor{bar_yellow1}{rgb}{.98, .98, 0.4}

\definecolor{cbg_red1}{rgb}{1, 0.85, 0.85}
\definecolor{bar_red1}{rgb}{1, 0.75, 0.85}

\definecolor{cbg_blue1}{rgb}{0.87843, 0.95686, 1.0}
\definecolor{bar_blue1}{rgb}{0.7,     0.95686, 1}

\usepackage{minted}
\usemintedstyle{autumn}

\usepackage[T1]{fontenc}
%\usepackage[latin1]{inputenc}
\usepackage{ucs}
\usepackage[utf8x]{inputenc}

% Set helvetica as the default font family:
\RequirePackage{helvet}
\renewcommand\familydefault{phv}

\usepackage{lmodern}         % Latin Modern fonts derived from Computer Modern

% Hyperlinks in PDF:
\definecolor{linkcolor}{rgb}{0,0,0.4}
\usepackage{hyperref}
\hypersetup{
    breaklinks=true,
    colorlinks=true,
    linkcolor=linkcolor,
    urlcolor=linkcolor,
    citecolor=black,
    filecolor=black,
    %filecolor=blue,
    pdfmenubar=true,
    pdftoolbar=true,
    bookmarksdepth=3   % Uncomment (and tweak) for PDF bookmarks with more levels than the TOC
    }
%\hyperbaseurl{}   % hyperlinks are relative to this root

\setcounter{tocdepth}{1}  % number chapter, section, subsection

% prevent orhpans and widows
\clubpenalty = 10000
\widowpenalty = 10000

% Make sure blank even-numbered pages before new chapters are
% totally blank with no header
\newcommand{\clearemptydoublepage}{\clearpage{\pagestyle{empty}\cleardoublepage}}
%\let\cleardoublepage\clearemptydoublepage % caused error in the toc

% --- end of standard preamble for documents ---


% insert custom LaTeX commands...

\raggedbottom
\makeindex

%-------------------- end preamble ----------------------

\begin{document}

% endif for #ifdef PREAMBLE
% #endif


% ------------------- main content ----------------------



% ----------------- title -------------------------

\thispagestyle{empty}
\hbox{\ \ }
\vfill
\begin{center}
{\huge{\bfseries{
\begin{spacing}{1.25}
{\rule{\linewidth}{0.5mm}} \\[0.4cm]
{DocOnce}
\\[0.4cm] {\rule{\linewidth}{0.5mm}} \\[1.5cm]
\end{spacing}
}}}

% ----------------- author(s) -------------------------

\vspace{0.5cm}

{\Large\textsf{Michael Weiß${}^{}$}}\\ [3mm]

{\Large\textsf{Simon Schäfer${}^{}$}}\\ [3mm]

\ \\ [2mm]

% ----------------- end author(s) -------------------------

% --- begin date ---
\ \\ [10mm]
{\large\textsf{Dec 11, 2015}}

\end{center}
% --- end date ---
\vfill
\clearpage

\tableofcontents


\vspace{1cm} % after toc




\chapter{DocOnce, eine universelle Markup-Sprache}
\section{Was ist DocOnce}
Doconce ist eine minimal ausgezeichnete Markup-Sprache, die alle wichtigen Funktionen einer Markup-Sprache bereitstellt und hierbei trotzdem die Anzahl der verwendeten Auszeichnungen möglichst gering hält. Durch die Verwendung von Doconce ist eine Konvertierung der Ausgangsdatei in viele unterschiedliche Formate (Latex, Markdown, PDF, ...) möglich. Außerdem lässt sich das Ausgabeformat für mehrere Anwendungsbereiche anpassen. Dies funktioniert für analoge Medien, wie z.B. Bücher, Abschlussarbeiten und wissenschaftliche Berichte, aber auch für digitale Veröffentlichungen, wie z.B. in Blogs oder auf Wiki-Seiten.
\section{Vor- und Nachteile}
\subsection{Vorteile}
\begin{itemize}
\item viele Ausgabeformate

\item geeignet für Autoren die für viele unterschiedliche Medien schreiben

\item Viele unterschiedliche Designs für das Erstellen von z.B. html-Dokumenten

\item Source-Code kann direkt aus Dateien ausgelesen werden

\item kann auch nur als einfacher Text ausgegeben werden z.B. für Email oder Code-Dokumentationen]
\end{itemize}

\noindent
\subsection{Nachteile}
\begin{itemize}
 \item Installation auf Windows nicht möglich

 \item erstellte Formate (z.B. .tex) benötigen (stellenweise) händische Überprüfung

 \item erbt probleme der Zielformate wie z.B. automatisches Setzen von Grafiken in LaTex
\end{itemize}

\noindent
\chapter{Installation}
\section{Ubuntu}
Die Installation unter Ubuntu kann durch ein Bash-Skript (siehe Anhang) durchgeführt werden. Zusätzlich zu den im Original Install-Skript angegebenen Pakete mussten, für die Installation des lxml-Python-Packages, auf Ubuntu 15.04 noch die development-packages  \texttt{libxml2-dev}, \texttt{libxslt1-dev} und \texttt{python-dev} installiert werden, da das Compilieren des Package ansonsten fehlschlug.

Nach der Installation kann DocOnce direkt im Terminal mit dem Befehl \texttt{doconce} und verschiedenen Parametern(siehe Kapitel 5 "Parameter für den \texttt{doconce}-Befehl") gestartet werden.
\chapter{Syntax}
\chapter{Ausgabeformate}

\section{Markdown}
\paragraph{ Verwendung von Markdown.}
\paragraph{ Ausgabeoptionen für Markdown.}
\paragraph{ Konvertierung von Markdown in weitere Formate.}
\section{Rest und Sphinx}
\paragraph{ Verwendung von Rest und Sphinx.}
\paragraph{ Ausgabeoptionen für Rest und Sphinx.}
\paragraph{ Konvertierung von Rest und Sphinx in weitere Formate.}
\section{LaTex}
\paragraph{ Verwendung von LaTex.}
\paragraph{ Ausgabeoptionen für LaTex.}
\paragraph{ Konvertierung von LaTex in weitere Formate.}
\section{HTML}
\paragraph{ Verwendung von HTML.}
\paragraph{ Ausgabeoptionen für HTML.}
\paragraph{ Konvertierung von HTML in weitere Formate.}
\section{Wiki-Formate}
\paragraph{ Verwendung von Wiki-Formaten.}
\paragraph{ Ausgabeoptionen für Wiki-Formate.}
\paragraph{ Konvertierung von Wiki-Formaten in weitere Formate.}
\chapter{Parameter für den \texttt{doconce}-Befehl}
\textbf{10\%}
\chapter{Fazit}
\textbf{10\%}
\chapter{Anhang}
\section{Setup-Script}
\begin{minted}[fontsize=\fontsize{9pt}{9pt},linenos=false,mathescape,baselinestretch=1.0,fontfamily=tt,xleftmargin=2mm]{bash}
#!/bin/bash
# Automatically generated script by deb2sh.py.

# The script is based on packages listed in debpkg_doconce.txt.

set -x  # make sure each command is printed in the terminal

function apt_install {
  sudo apt-get -y install $1
  if [ $? -ne 0 ]; then
    echo "could not install $1 - abort"
    exit 1
  fi
}

function pip_install {
  sudo pip install --upgrade "$@"
  if [ $? -ne 0 ]; then
    echo "could not install $p - abort"
    exit 1
  fi
}

sudo apt-get update --fix-missing

# Installation script for doconce and all dependencies

# This script is translated from
# doc/src/manual/debpkg_doconce.txt
# in the doconce source tree, with help of
# vagrantbox/doc/src/vagrant/src-vagrant/deb2sh.py
# (git clone git@github.com:hplgit/vagrantbox.git)

# Python v2.7 must be installed (doconce does not work with v3.x)
pyversion=`python -c 'import sys; print sys.version[:3]'`
if [ $pyversion != '2.7' ]; then echo "Python v${pyversion} cannot be used with DocOnce"; exit 1; fi

# Install downloaded source code in subdirectory srclib
if [ ! -d srclib ]; then mkdir srclib; fi

# Version control systems
apt_install mercurial
apt_install git
apt_install subversion

# --- Python-based packages and tools ---
apt_install python-pip
apt_install idle
apt_install python-dev
apt_install python-pdftools
pip_install ipython --upgrade
pip_install tornado --upgrade
pip_install pyzmq --upgrade
pip_install traitlets --upgrade
pip_install pickleshare --upgrade
pip_install jsonschema
# If problems with IPython.nbformat.v4: clone ipython and run setup.py
# to get the latest version

# Preprocessors
pip_install future
pip_install mako --upgrade
pip_install -e git+https://github.com/hplgit/preprocess#egg=preprocess --upgrade

# Publish for handling bibliography
pip_install python-Levenshtein
pip_install lxml
pip_install -e hg+https://bitbucket.org/logg/publish#egg=publish --upgrade

# Sphinx (with additional third/party themes)
pip_install sphinx

pip_install alabaster --upgrade
pip_install sphinx_rtd_theme --upgrade
pip_install -e hg+https://bitbucket.org/ecollins/cloud_sptheme#egg=cloud_sptheme --upgrade
pip_install -e git+https://github.com/ryan-roemer/sphinx-bootstrap-theme#egg=sphinx-bootstrap-theme --upgrade
pip_install -e hg+https://bitbucket.org/miiton/sphinxjp.themes.solarized#egg=sphinxjp.themes.solarized --upgrade
pip_install -e git+https://github.com/shkumagai/sphinxjp.themes.impressjs#egg=sphinxjp.themes.impressjs --upgrade
pip_install -e git+https://github.com/kriskda/sphinx-sagecell#egg=sphinx-sagecell --upgrade

# Runestone sphinx books
pip_install sphinxcontrib-paverutils
pip_install paver
pip_install cogapp

#pip install -e git+https://bitbucket.org/sanguineturtle/pygments-ipython-console#egg=pygments-ipython-console
pip_install -e git+https://bitbucket.org/hplbit/pygments-ipython-console#egg=pygments-ipython-console
pip_install -e git+https://github.com/hplgit/pygments-doconce#egg=pygments-doconce

# XHTML
pip_install beautifulsoup4
pip_install html5lib

# ptex2tex is not needed if --latex_code_style= option is used
cd srclib
svn checkout http://ptex2tex.googlecode.com/svn/trunk/ ptex2tex
cd ptex2tex
sudo python setup.py install
cd latex
sh cp2texmf.sh
cd ../../..

# LaTeX
apt_install texinfo
apt_install texlive-full
#apt_install texlive-extra-utils
#apt_install texlive-latex-extra
#apt_install texlive-latex-recommended
#apt_install texlive-math-extra
#apt_install texlive-font-utils
#apt_install texlive-humanities
apt_install latexdiff
apt_install auctex

# Image manipulation
apt_install imagemagick
apt_install netpbm
apt_install mjpegtools
apt_install pdftk
apt_install giftrans
apt_install gv
apt_install evince
apt_install smpeg-plaympeg
apt_install mplayer
apt_install totem
apt_install libav-tools

# Misc
apt_install ispell
apt_install pandoc
apt_install libreoffice
apt_install unoconv
apt_install libreoffice-dmaths
#epydoc is an old-fashioned output format, will any doconce user use it?
#pip install -e svn+https://epydoc.svn.sourceforge.net/svnroot/epydoc/trunk/epydoc#egg=epydoc

apt_install curl
apt_install a2ps
apt_install wdiff
apt_install meld
apt_install diffpdf
apt_install kdiff3
apt_install diffuse

# tkdiff.tcl:
#tcl8.5-dev tk8.5-dev blt-dev
#https://sourceforge.net/projects/tkdiff/

# example on installing mdframed.sty manually (it exists in texlive,
# but sometimes needs to be in its newest version)
git clone https://github.com/marcodaniel/mdframed
if [ -d mdframed ]; then cd mdframed; make localinstall; cd ..; fi
#$ echo "remove the mdframe directory (if successful install of mdframed.sty): rm -rf mdframed"

# DocOnce itself
cd srclib
git clone https://github.com/hplgit/doconce.git
if [ -d doconce ]; then cd doconce; sudo python setup.py install; cd ../..; fi
echo "Everything is successfully installed!"
\end{minted}

% ------------------- end of main content ---------------


% #ifdef PREAMBLE
\clearemptydoublepage
\markboth{Index}{Index}
\thispagestyle{empty}
\printindex

\end{document}
% #endif

